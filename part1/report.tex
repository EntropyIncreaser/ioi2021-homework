%!TEX TS-program = xelatex
%!TEX encoding = UTF-8

% XeLaTeX can use any Mac OS X font. See the setromanfont command below.
% Input to XeLaTeX is full Unicode, so Unicode characters can be typed directly into the source.

% The next lines tell TeXShop to typeset with xelatex, and to open and save the source with Unicode encoding.

%!TEX TS-program = xelatex
%!TEX encoding = UTF-8 Unicode

\documentclass[12pt]{ctexart}
\usepackage{geometry}                % See geometry.pdf to learn the layout options. There are lots.
\geometry{a4paper}                   % ... or a4paper or a5paper or ... 
%\geometry{landscape}                % Activate for for rotated page geometry
%\usepackage[parfill]{parskip}    % Activate to begin paragraphs with an empty line rather than an indent
\usepackage{graphicx}
\usepackage{amssymb}
\usepackage{tabularx}
\usepackage{amsmath}
\usepackage{mathrsfs}
\usepackage{listings}
\usepackage{color}
\usepackage{hyperref}
\usepackage{amsthm}
\usepackage{chemfig}
\usepackage{mhchem}
\usepackage{titlesec}
\usepackage{tcolorbox}
\usepackage{fancyhdr}
\usepackage{tikz-cd}

\tcbuselibrary{skins, breakable, theorems}

% Will Robertson's fontspec.sty can be used to simplify font choices.
% To experiment, open /Applications/Font Book to examine the fonts provided on Mac OS X,
% and change "Hoefler Text" to any of these choices.

\usepackage{fontspec,xltxtra,xunicode}

\providecommand*{\unit}[1]{\ensuremath{\mathrm{\,#1}}}
\newcommand{\di}{\ensuremath{\mathrm{\,d}}}

% \setsansfont{TeX Gyre Heros}

\newfontfamily\sfsf{TeX Gyre Heros}

\setCJKmainfont[BoldFont=Noto Serif CJK SC Black]{Noto Serif CJK SC}
\setCJKsansfont[BoldFont=Noto Sans CJK SC Black]{Noto Sans CJK SC}

\setCJKfamilyfont{emfont}[
	BoldFont=PingFangSC-Medium
]{PingFangSC-Regular}
\renewcommand{\em}{\bfseries\sffamily\CJKfamily{emfont}} % 强调

% \CTEXsetup[format={\large\bfseries}]{section}

\titleformat{\section}
{\centering\sffamily\Large\bfseries}{\sfsf{{\thesection}}}{1em}{}

\titleformat{\subsection}
{\sffamily\large\bfseries}{\sfsf{{\thesubsection}}}{1em}{}

\titleformat{\subsubsection}
{\sffamily\bfseries}{\sfsf{{\thesubsubsection}}}{1em}{}

\newtcolorbox{source}{
	breakable,
	enhanced,
	width = \textwidth,
	colback = white, colbacktitle = white,
	colframe = black, boxrule=0.2mm,
	coltitle = black,
	fonttitle = \sffamily,
	attach boxed title to top left = {yshift=-\tcboxedtitleheight/2,  xshift=\tcboxedtitlewidth/4},
	boxed title style = {boxrule=0pt, colframe=white},
	before skip = 0.5cm,
	top = 4mm,
	bottom = 3mm,
	title={来源}
}

\hypersetup{colorlinks = true,
			linkcolor = blue,
			citecolor = red,
			urlcolor = teal}

\pagestyle{fancy}
\fancyhf{}
\fancyhead[C]{\leftmark}
\fancyfoot[C]{\thepage}

\linespread{1.5}
\zihao{4}

\title{\textbf{\Huge 解题报告\\
\large IOI2021 国家集训队第一阶段作业第一部分}}
\author{\textbf{北京大学附属中学\quad 李白天}}
\date{}

\begin{document}

\maketitle

\tableofcontents

\newpage

\section{Fygon 2.0}

\begin{source}\center
2017-2018 ACM-ICPC, NEERC

Northern Subregional Contest (\href{https://codeforces.com/gym/101612/}{\texttt{gym101612F}})
\end{source}

\subsection{题目大意}

\newcommand{\for}{\texttt{for}}
\newcommand{\lag}{\texttt{lag}}
\providecommand*{\var}[1]{{%\color{magenta}
\texttt{<#1>}}}

\theoremstyle{definition}
\newtheorem*{example}{\bfseries\textsf {例}}

给一个仅由 \for 循环和 \lag 操作构成的程序,且保证程序仅为一路缩进到底的嵌套 \for 循环结构,最内层为一个单一的 \lag 操作。详言之,一个 \for 循环会声明一个整型变量 \var{variable} 且保证与外部变量不重名,该变量将从 \var{from} 迭代至 \var{to}(若 \var{from} 的值大于 \var{to} 则不会执行),其中 \var{from} 可以取 $1$ 或者外部循环定义的变量,\var{to} 可以取变量 \texttt{n}(一个输入的量)或者外部定义的变量。

记 $f(n)$ 为给定程序执行的 \lag 次数,关于 $n$ 的函数。试求出非负整数 $k$ 和正有理数 $C$,满足

$$
\lim_{n\rightarrow \infty} \frac{f(n)}{C\cdot n^k} = 1
$$

其中 $C$ 的输出方式为输出其既约分数 $C=p/q, \gcd(p,q)=1$ 的 $p,q$。

\subsection{数据范围}

记所给的 \for 循环数量为 $k$,保证 $k\le 20$。

\subsection{解题过程}

首先,让我们将问题进行重新表述。记第 $j$ 个循环的变量为 $x_j$,根据 \for 循环的限制可知,我们所求的 $f(n)$ 就是对于每个循环变量,对于其各自的 $\var{from}_j$ 和 $\var{to}_j$,满足 $\var{from}_j \le x_j \le \var{to}_j$ 的全体整数解 $(x_1, \dots, x_k)$ 的数量。

为了看清这个问题,我们不妨先考虑这样一个简单的例子:

\begin{example}
有一个 $3$ 个 \for 循环组成的程序,变量约束为 $1\le x_1\le n, 1\le x_2\le x_1, x_1 \le x_3\le x_2$。在这个例子中我们可以看到,因为 $x_1\le x_2$ 且 $x_2\le x_1$,所以必然有 $x_1=x_2$ 的时候才可能进入执行 \lag;进一步地,又因为 $x_1\le x_3 \le x_2=x_1$,我们有 $x_3=x_2=x_1$。由此我们看到,在这个例子中虽然有三个变量,实则只有一个自由度。执行次数 $f(n) = n$,因此 $C=1,k=1$。
\end{example}

上述例子启发我们用图论的视角来考察本问题。我们考虑由问题得到的若干变量之间的序关系转化为一个 $k$ 个节点的\emph{有向图} $G$,不妨将节点编号从 $1$ 至 $k$。若 \for 循环给出了 $x_i \le x_j$ 的限制,则将 $i$ 向 $j$ 连一条边。

在图 $G$ 中,若 $i$ 可达 $j$,则说明存在一条路径 $i \rightarrow \cdots \rightarrow j$,对应于 \for 循环中发掘出的序关系的一条链 $x_i \le \cdots \le x_j$,根据序关系的传递性,可知 $x_i\le x_j$。

我们再考虑图中的一个强连通分量 $S$,对于 $i,j\in S$,由强连通分量的定义可知 $i,j$ 是互相可达的,因此由上述分析可知 $x_i\le x_j$ 且 $x_j\le x_i$,由序关系的反对称性可知 $x_i=x_j$。

由此可知,我们考虑将 $G$ 进行\emph{缩点}得到的图 $G^{\star}$,即 $G^{\star}$ 中的每个点对应于 $G$ 中的一个强连通分量,$G^{\star}$ 中的节点 $u^\star$ 向 $v^\star$ 连边 $(u^\star \neq v^\star)$,当且仅当对应于 $G$ 中的强连通分量 $S,T$ 存在 $i\in S,j\in T$ 且 $(i,j)\in G$。我们不妨设 $|G^\star| = t$,其节点从 $1^\star$ 编号至 $t^\star$。缩点后得到的图具有明确的实际意义,即我们不妨设为变量 $y_1, \dots, y_t$,则 $y_u$ 对应等于 $u^\star$ 在 $G$ 中所对应强连通分量的每一个变量(他们取值总是相同的)。即 $\{y_u\}$ 是 $\{x_i\}$ 的一组简化情况,且因为 $G^\star$ 是\emph{有向无环图},我们有一个直觉性的事情是 $\{y_u\}$ 这组变量是具有 $t$ 个\emph{自由度}的。我们不妨考虑有向无环图的一个充要条件,即\emph{存在拓扑排序}。我们注意到,任取 $G^\star$ 的一个拓扑序 $\{p_u\}$,其与 $\binom n t$ 个满足 $y_u$ 互不相同的解一一对应。我们不难得出对应方式:

\begin{itemize}
\item 对于一个拓扑序 $p_1, \dots, p_t$ 满足对任意 $1\le u<v\le t$,$p_v$ 向 $p_u$ 没有边。那么我们任取 $\{1,\dots,n\}$ 的一组 $t$ 元子集,按照从小到大的方式赋予 $y_{p_1},\dots,y_{p_t}$,即得到了一组解,由拓扑排序的性质可知其合法。
\item 对于任意一组 $y$ 值互不相同的解,我们设其大小关系为 $y_{p_1}<\dots<y_{p_t}$,那么由其合法性可知 $p_1, \dots, p_t$ 是一个合法的拓扑序,且 $y$ 的取值构成的集合对应于 $\{1,\dots,n\}$ 的一组 $t$ 元子集。
\end{itemize}

由前述分析,我们可知如果 $G^\star$ 有 $K$ 种拓扑序,则存在至少 $K \cdot \binom n t$ 种解。我们考虑分析 $\lim_{n\rightarrow \infty} \frac{\binom n t}{n^t}$,由于 $\binom nt = \frac{n(n-1)\cdots(n-t+1)}{t!}$ 是 $t$ 次多项式,我们可知这一极限是一非零常数,且该常数等于该多项式的 $t$ 次项,也即 $\frac 1{t!}$。

而对于可能有数相等的情况呢?直觉上说由于其减小了自由度,是对我们所需的 $C,k$ 没有影响的。形式化地,我们考虑只要添加了若干个相等条件,那所剩的方案就可以进一步缩减为 $<k$ 个变量的相异整数解数。设其方案数是 $<k$ 次的多项式 $g(n)$,因此必有 $\lim_{n\rightarrow \infty} \frac {g(n)}{n^t}=0$。而我们添加的相等条件的总共方法数量虽然庞大,却只与 $k$ 有关,是有限多的,因此其总和除以 $n^t$,在 $n$ 趋于无穷大时仍然为 $0$。综上所述,该极限仅由互不相同的整数解贡献,且具体值为 $C=\frac K{t!}$。

由 $K$ 的定义可知其是整数且 $1\le K\le t!$,而 $t\le k\le 20$,故 $t! \leq 20! \approx 2.43\times 10^{18}$,故该分子分母可以通过 64 位整型存储,通过欧几里得算法计算出二者的最大公约数并约分。

接下来所剩的便是算法问题了,其一为如何得到 $G^\star$,其二为如何进一步计算出 $K$。

对于求出 $G^\star$,我们可以使用广为人知的 Tarjan 算法得到图的各强连通分量,并进一步根据得到的强连通分量按照前述定义建出 $G^\star$,由于 $G$ 的点数为 $k$,边数不超过 $2k$,这部分的复杂度为 $\Theta(k)$。

接下来求算 $K$ 的部分为算法的瓶颈,我们考虑进行状态压缩动态规划,设 $f(S)$ 为 $G^\star$ 的导出子图 $S$ 的拓扑序数量 $(S\subseteq G^\star)$。立得 $f(\varnothing)=1$,而对于 $S$ 非空的情况,考虑接下来拓扑排序的第一个点,若 $u^\star$ 在 $S$ 的导出子图中没有入边,则可以作为第一个点。我们记入点集 $I(u^\star)=\{v^\star \vert (v^\star,u^\star)\in G^\star\}$,则可得对于 $S\neq \varnothing$,有转移

$$
f(S) = \sum_{u^\star \in S\wedge I(u^\star) \cap S=\varnothing} f(S\backslash \{u^\star\})
$$

其中,入度判据在实现时可以通过位运算在常数时间内进行判定,总共有 $2^t$ 个状态,每个状态只需 $\Theta(t)$ 的时间对 $u^\star$ 进行枚举,故这一部分的复杂度为 $\Theta(t2^t)$。

综上,本题在 $t=k$,即 $G$ 不发生退化的情况下达到最坏复杂度,为 $\Theta(k2^k)$,而题目限制中满足 $k\le 20$,是完全可以接受的。事实上,有向无环图的拓扑排序计数问题已经被证明是 \emph{\#P-Complete 问题}\cite{sharppc},因此我们也可以基本满足于指数级复杂度的算法了。

\newpage

\section{Evolution in Parallel}

\begin{source}\center
2015 ACM-ICPC World Finals (\href{https://codeforces.com/gym/101239/}{\texttt{gym101239E}})
\end{source}

\subsection{题目大意}

给出 $n$ 个字符串 $S_1, \dots, S_n$ 和一个目标串 $T$,设 $S_0=\varnothing$ 为空串以及 $S_{n+1}=T$,问能否找出两条以 $0$ 为起点,以 $n+1$ 为终点的路线,满足:

\begin{itemize}
\item 他们仅在起终点处相交,且 $1,\dots,n$ 每个点均在一条路径中,即 $p_1,\dots,p_{s_1}$ 和 $q_1,\dots,q_{s_2}$ 给出了 $\{1,\dots,n\}$ 的一个划分 $(p_0=q_0=0, p_{s_1+1}=q_{s_2+1}=n+1)$。
\item 路径中的每个点所代表的串为路径下一个点的\emph{严格子序列},即可以通过插入正整数个字符来得到对应字符串。
\end{itemize}

如有解,则给出一组解。

\subsection{数据范围}

保证 $1\le n, |S_i|, |T|\le 4\times 10^3$。字符集为 $\Sigma = \{\texttt{A},\texttt{C},\texttt{M}\}$。输入的所有串互不相同。

\subsection{解题过程}

首先让我们来整理一下子序列关系的性质,接下来我们记 $S \prec T$ 当且仅当 $S$ 是 $T$ 的一个严格子序列。我们容易发现这一关系具有\emph{传递性}:若 $S\prec T$ 且 $T\prec R$,那么 $S\prec R$,根据定义我们只需将 $T$ 添加得到 $R$ 的那些字符按顺序拼接在 $S$ 添加得到 $T$ 的过程之后,即可验证这一性质。

接下来,我们不难注意到 $S\prec T$ 有一个必要条件:$|S|<|T|$,因为 $T$ 总是由 $S$ 添加字符得到的。那么我们不妨将 $S_i$ 按照长度从小到大排列,那么接下来就满足一个性质:$S_i \prec S_j \Rightarrow i < j$。此时序列 $p,q$ 只能是分别单调递增的。

接下来我们假设恰好规划好了 $1\sim k$,需要确认依此前提能否得到一个解。需注意到一个虽平凡却重要的事实:此时 $p,q$ 必有一者的尾部为 $k$。接着不妨设另一者为 $t$,我们首先来尝试讨论将 $k+1$ 加入哪条路径,我们可进行分类讨论:

\begin{enumerate}
\item $S_t \nprec S_{k+1} \wedge S_k \nprec S_{k+1}$,这种情况下只能是无解的,因为二者均不可能经过 $k+1$。
\item $S_t \prec S_{k+1}, S_k \prec S_{k+1}$ 恰有一者成立,此时我们分配 $k+1$ 的上一个节点的方案是唯一的,所以只能选择对应方案,进而规约到 $k+1$ 前缀对应的问题。
\item $S_t \prec S_{k+1} \wedge S_k\prec S_{k+1}$ 即同时成立,此时的情况是较为复杂的,我们无法在此时断言由何者连向 $k+1$ 是更正确的,因此我们需要考虑更往后的字符串。
\end{enumerate}

接下来我们需要详细讨论上述的第 3 种情形,我们考虑最大的 $r$,满足关系 $S_k \prec S_{k+1}\prec \cdots \prec S_r$。若 $r=n+1$ 此时我们已经得到了一组解,接下来考虑非平凡的情况。此时依 $r$ 的最大性有 $S_r \nprec S_{r+1}$,因此 $r,r+1$ 应各属一条路径,且 $r$ 和 $r+1$ 此时各为一个结尾,因此我们知道有解情况被转化成判断唯一一种 $r+1$ 前缀的情况是否有解。而在判断其是否有解之前,我们还需明确 $k+1 \sim r+1$ 一族该如何规划路径。读者可端详已将已有条件展示清楚的下图:

\begin{center}
% https://tikzcd.yichuanshen.de/#N4Igdg9gJgpgziAXAbVABwnAlgFyxMJZABgBoBGAXVJADcBDAGwFcYkRiQBfU9TXfIRQAmUsWp0mrdjm68QGbHgJFRwiQxZtEIANZy+SwUQDMFDVO16A1OQML+yocgCs5mpuk6ATvcUCVFAA2d0ktdm9bbgkYKABzeCJQADNvCABbJDMQHAgkABYPS3YAHRK0bxgAYwACMqqoCBw4MorqkBpGLDArKHo4AAtY+1SMpHIaXKyi8J1WyqqRtMzEURy8xGzPK3n2zvoAIxhGAAVHYx8sOIHZHhTlpDJ18ZojsChHma8QXar6xuadXKCw6IC6PXYfUGwxoQ3oHx0kAhk3oWEY7CRbH2R1O50CYJgyVu8lGKyeU1W2OOZyM+O8VxuoO2pWB1X+TRarMWnW6vX6Qw+dxApJez0QbhAbwRJiezJ0AH4lmNKWKJVKkABaGVU3G0oQgenXWRfKyKriULhAA
\begin{tikzcd}
                                                                                             &  & t \arrow[rd, "\prec"] \arrow[rrrrd, "?", bend left]    &                                             &  &   &     \\
0 \arrow[rru, "\prec\cdots \prec", no head, dashed] \arrow[rrd, "\prec\cdots\prec"', dashed] &  &                                                        & k+1 \arrow[rr, "\prec \cdots\prec", dashed] &  & r & r+1 \\
                                                                                             &  & k \arrow[ru, "\prec"'] \arrow[rrrru, "?"', bend right] &                                             &  &   &    
\end{tikzcd}
\end{center}

可以发现,若 $S_t\nprec S_{r+1} \wedge S_k \nprec S_{r+1}$,由于二者已经占据了两条路径,此时必然无解,否则只需将其中任何符合子序列条件的一者连接到 $r+1$,并将另一者连接到 $k+1$,就完成了这一部分的构造。

由此,我们可以将上述判断方式改写为算法。由于每一步要么得到无解,要么得到唯一的后继情况,我们在实现时实际上只需写成迭代的形式。

对于判定 $S_i$ 是否是 $S_j$ 的一个子序列,我们可以采取贪心算法,我们考虑维护对于 $S_j$ 的前 $k$ 个字符构成的前缀,$S_i$ 的最大前缀长度使得该前缀是这 $S_j$ 的该前缀的子序列。当 $k$ 增大的时候,只需判定对应的前缀能否增大 $1$。由于我们只需在 $|S_i|<|S_j|$ 的时候运行这一过程,因此复杂度是 $\Theta(|S_j|)$ 的。

为了分析所有的比较所引发的复杂度,我们仔细分析上述的迭代过程,注意到在进行分类讨论的归类情况中,会有两个串与 $S_{k+1}$ 进行比较,而在第三类的情况中,对于 $S_{r+1}$ 总共与前面的三个串 $S_t,S_k,S_{r+1}$ 进行了比较。因此综上所述,每个串作为较长串比较的次数不超过 $3$ 次。比较所消耗的总共复杂度为 $\Theta(\sum |S_i|)$,即总共的串长。

而余下的其他部分皆非瓶颈。最初所进行的排序如使用基数排序便可做到 $\Theta(\max |S_i|)$,最后的检验复杂度为 $\Theta(|T|)$。综上所述,本题在 $\Theta(\sum |S_i| + |T|)$ 的时间内得到解决,由于这已经达到了输入下界,是一个优秀的算法。

\newpage

\section{Interactive Interception}

\begin{source}\center
2013-2014 ACM-ICPC, NEERC

Northeastern European Regional Contest (\href{https://codeforces.com/gym/100307}{\texttt{gym100307I}})
\end{source}

\subsection{题目大意}

这是一道交互题。

已知数轴上有一点在进行匀速运动。初始给定正整数 $p,v$,表示该点位于一整点 $x$ 满足 $x \in [0,p]$,且具有整数速度 $q$ 满足 $q\in [0,v]$。程序需要最晚在第 $100$ 个时刻确认此时刻的该点所在位置。

从 $0$ 时刻开始每个时刻,程序可以给出一个区间 $[L, R](0\le L\le R\le 10^9)$,交互器将返回此时该点的位置是否位于 $[L,R]$ 内部,收到消息之后进入下一时刻,且该点的位置增加 $q$。

\subsection{数据范围}

保证 $1\le p,v\le 10^5$。

\subsection{解题过程}

由于可以询问的范围是一个区间,这启发我们通过二分查找来确定点的位置和速度。在最初,我们若令 $L=0, R=\lfloor \frac p 2 \rfloor$,则询问结果能帮助我们直接排除初始位置中一半的可能性。若正确则说明整点落在 $[0, \lfloor \frac p 2 \rfloor]$ 内。而错误则说明整点落在了 $[\lfloor \frac p 2\rfloor + 1, p]$ 内。但当进入下一个时刻的时候便麻烦了:设上一时刻点在区间 $[l, r]$ 内,由于该点的速度尚未确定,假设速度为 $0$,那位置还是落在 $[l, r]$ 内。但如果速度为 $1$,就会落在 $[l+1,r+1]$ 内……实际上我们能确定的候选位置,仅仅是这些情况的并。即此时我们只能断言位置在 $[l, r+v]$ 内。但是注意到在下一次询问得到回答后,我们又可以反过来推断速度的可能区间。具体地,我们假设当前位于时刻 $t$,若已知在 $k(0\le k\le t)$ 时刻,点的位置在区间 $[l_k, r_k]$ 内,那么由题设有不等式

$$
l_k \le x + qk\le r_k
$$

而任取 $j<k$ 整理不等式可得

\begin{align}
x+kq & \le r_k \nonumber  \\
-x-jq & \le -l_j \nonumber \\
(k-j) q & \le r_k - l_j \nonumber \\
q & \le \frac {r_k - l_j}{k-j} \label{boundq}
\end{align}

对称地,由 $x+kq\ge l_k$ 和 $-x-jq \ge -r_j$ 可得 $q\ge  \frac{l_k -r_j}{k-j}$。

因此,当 $t$ 时刻的位置区间 $[l_t, r_t]$ 得到确定之后,我们由 $j<k=t$ 的情况得到了新的 $2(t-1)$ 组不等式,从而进一步限制速度区间 $[q_l, q_r]$。

而在进行 $t$ 时刻的询问的时候,我们同样需要将以前的所有信息纳入考虑,可得

\begin{align*}
x + jq & \ge l_j\\
x + tq & \ge l_j + (t-j)q\\
 & \ge l_j + (t-j)q_l
\end{align*}

对称地,我们可以得到 $x + tq \le r_j + (t-j)q_r$。这些限制就限定出了当前的位置区间 $[x_l,x_r]$,我们取 $x_m=\lfloor \frac {x_l+x_r}2\rfloor$ 作为中点,即可询问点在是否在区间 $[x_l, x_m]$,若不是则说明在 $[x_m+1,x_r]$,如此一来便将候选位置缩小一半了。

而在这一过程中,候选的区间每次是随下一个时刻重新确定,然后减半,往复循环,在每次重新确定的过程中,区间是有可能再次变大的。这个过程的实际复杂度还需我们严谨确认。

为了方便分析,我们不妨将问题适当放宽:假设最初的位置和速度分别是在 $[0,p]$ 和 $[0,v]$ 内的实数,而我们每次询问的范围也是实数区间,当我们某个时刻确认的位置区间长度 $<1$ 时则算成功。

容易发现这个转化是严格不弱于原问题的,因为当确认的区间长度 $<1$ 时仅有一个整点,而询问时只需将区间的左端向上取整,右端向下取整。

接下来我们考虑两个量, $P_t$ 表示在 $t$ 时刻的询问完成后确认的位置区间长度,$V_t$ 表示在 $t$ 时刻的询问完成后确认的速度区间长度。依题设我们可知 $V_0=v$,而第一步二分使得 $P_0= \frac p2$。

考虑 $P_t$ 的一个上界,任取 $j<t$,由于在确认区间的时候会对其取交,而 $j$ 时刻得到的区间在 $t$ 时刻的长度为 $P_j+(t-j)V_{t-1}$,因此我们得到

\begin{equation}
P_t \le \frac{P_j + (t-j)V_{t-1}}2 \label{boundp}
\end{equation}

接着,我们考虑 $V_t$ 的上界。不妨设在 $t$ 时刻最终确认的区间为 $[l_t,r_t]$,由 \eqref{boundq} 及其对称式子可得,

\begin{align}
q_r &\le \frac {r_t - l_j}{t-j}\nonumber\\
q_l &\ge \frac {l_t - r_j}{t-j}\nonumber\\
V_t &= q_r-q_l\nonumber\\
 &\le \frac {r_t - l_t + r_j - l_j}{t-j}\nonumber\\
 &= \frac{P_t + P_j}{t-j} \label{boundv}
\end{align}

由此,\eqref{boundp} 和 \eqref{boundv} 整体构成了一组交错影响的不等式关系。若设 $p,v\le N$,依其逐项递推可得到系数 $x_t, y_t$ 满足 $P_t \le x_t N, V_t\le y_t N$。实际计算发现,在 $t=64$ 时有 $x_t \approx 4.7 \times 10^{-6}$。因此在题目给定的数据范围下,$2\cdot P_{64} \leq 2\cdot x_{64} \cdot 10^5 < 1$,故我们得到了一个上界,即询问最晚会在第 $64$ 个时刻结束。

于此同时,我们也可分析这一算法的渐进行为。注意到 \eqref{boundp} 和 \eqref{boundv} 这两组不等式均具有平移性质,即对于 $(t, j)$ 之间的不等式系数与 $(t+1,j+1)$ 之间的不等式系数是相同的,而涉及的变量相差一个平移。

由此可知,若存在常数 $d, \alpha (\alpha < 1)$ 使得 $P_d \le \alpha \cdot P_0$ 且 $V_d \le \alpha \cdot V_0$,那么对所有 $t\ge d$ 均有 $P_t \le \alpha P_{t-d}$。进而有 $P_t \le \alpha^{\lfloor t/d\rfloor} P_{t \bmod d} \le \alpha^{\lfloor t/d\rfloor} N$。由此可导出在不超过 $d(\log_{1/\alpha} N + O(1))$ 的时刻内即可确定位置,进而说明总共的询问次数为 $\Theta(\log N)$。

经计算发现,取 $d=5$ 时,$\alpha \approx 0.88$ 满足条件。由此可知上述算法的询问此时为 $O(\log \max (p, v))$,而由于中途还有一层枚举,所以程序执行的复杂度为 $O(\log^2 \max(p, v))$。

此外,注意到本题的答案空间有 $(p+1)\cdot (v+1)$ 种,而询问的结果只有 $2$ 种,因此从信息熵的角度可知,最优的询问策略在最坏情况下也至少需要 $\log_2 ((p+1)\cdot (v+1)) = O (\log \max (p, v))$ 次询问。从这方面考虑,在 $p,v$ 同阶时,本算法的询问复杂度已经达到最优。

\newpage

\begin{thebibliography}{}
\addcontentsline{toc}{section}{参考文献}
\bibitem{sharppc} Brightwell, Graham R.; Winkler, Peter (1991). "Counting linear extensions". Order. 8 (3): 225–242. doi:\href{https://doi.org/10.1007\%2FBF00383444}{\texttt{10.1007/BF00383444}}
\end{thebibliography}

\end{document}

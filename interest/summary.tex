
% XeLaTeX can use any Mac OS X font. See the setromanfont command below.
% Input to XeLaTeX is full Unicode, so Unicode characters can be typed directly into the source.

% The next lines tell TeXShop to typeset with xelatex, and to open and save the source with Unicode encoding.

%!TEX TS-program = xelatex
%!TEX encoding = UTF-8 Unicode

\documentclass[12pt]{ctexart}
\usepackage{geometry}                % See geometry.pdf to learn the layout options. There are lots.
\geometry{a4paper,left=1.5cm,right=1.5cm,top=2.5cm,bottom=2.5cm}                   % ... or a4paper or a5paper or ... 
%\geometry{landscape}                % Activate for for rotated page geometry
%\usepackage[parfill]{parskip}    % Activate to begin paragraphs with an empty line rather than an indent
\usepackage{graphicx}
\usepackage{amssymb}
\usepackage{tabularx}
\usepackage{amsmath}
\usepackage{mathrsfs}
\usepackage{listings}
\usepackage{color}
\usepackage{hyperref}
\usepackage{amsthm}
\usepackage{titlesec}
\usepackage{tcolorbox}
\usepackage{fancyhdr}
\usepackage{multirow}
\usepackage{longtable}

\tcbuselibrary{skins, breakable, theorems}

% Will Robertson's fontspec.sty can be used to simplify font choices.
% To experiment, open /Applications/Font Book to examine the fonts provided on Mac OS X,
% and change "Hoefler Text" to any of these choices.

\usepackage{fontspec,xltxtra,xunicode}

\providecommand*{\unit}[1]{\ensuremath{\mathrm{\,#1}}}
\newcommand{\di}{\ensuremath{\mathrm{\,d}}}
\newcommand{\Mul}{\ensuremath{\mathsf M}}
\newcommand{\me}{\ensuremath{\mathrm e}}


% \setsansfont{TeX Gyre Heros}

\newfontfamily\sfsf{TeX Gyre Heros}

\setCJKmainfont[BoldFont=Noto Serif CJK SC Black]{Noto Serif CJK SC}
\setCJKsansfont[BoldFont=Noto Sans CJK SC Black]{Noto Sans CJK SC}

\setCJKfamilyfont{emfont}[
	BoldFont=PingFangSC-Medium
]{PingFangSC-Regular}
\renewcommand{\em}{\bfseries\sffamily\CJKfamily{emfont}} % 强调

% \CTEXsetup[format={\large\bfseries}]{section}

\titleformat{\section}
{\centering\sffamily\Large\bfseries}{\sfsf{{\thesection}}}{1em}{}

\titleformat{\subsection}
{\sffamily\large\bfseries}{\sfsf{{\thesubsection}}}{1em}{}

\titleformat{\subsubsection}
{\sffamily\bfseries}{\sfsf{{\thesubsubsection}}}{1em}{}


\hypersetup{colorlinks = true,
			linkcolor = blue,
			citecolor = red,
			urlcolor = teal}

\pagestyle{fancy}
\fancyhf{}
\fancyhead[C]{\leftmark}
\fancyfoot[C]{\thepage}

\linespread{1.5}
\zihao{4}

\theoremstyle{theorem}
\newtheorem{lemma}{\bfseries\textsf {引理}}

\theoremstyle{theorem}
\newtheorem{theorem}{\bfseries\textsf {定理}}

\title{\textbf{IOI 2021 集训队自选题数数部分考察报告}}
\author{\textbf{Elegia}}
\date{}

\begin{document}

\maketitle

\tableofcontents

\newpage

\section{春天,在积雪下结一成形,抽枝发芽}

\paragraph{简要题意} 输入 $n$,对于 $m=1\sim n$,输出 $m$ 阶\emph{单排列}(不存在长为 $2\sim m-1$ 的连续段的排列)的数量。

同余 $998244353$,保证 $n\le 10^5$。

\paragraph{解法}

直观上看,单排列反应了某种排列的层级结构。我们考虑一个排列不停地缩掉一个非平凡的连续段,那么最后将会得到一个单排列。

但是这种说法并不准确,我们发现缩完的排列如果是形如 $[1,2]$ 或者 $[2,1]$ 的话,那么可能并不是唯一的方式,这取决于实际上这个排列被切割成了几段。

我们考虑记排列的 OGF 为 $F(x) = \sum_{n\ge 1} n!x^n$,单排列的 OGF 是 $G(x)$,前述的切割无法进行的排列的 OGF 是 $H(x)$,那么不难得到
\begin{align*}
F &= \sum_{n\ge 1}H^n\\
&= \frac{H}{1-H}\\
\frac 1F &= \frac 1H - 1
\end{align*}
接下来考虑 $G$ 的构成。首先去掉 $n=1$ 的情况,也即考虑 $(G - x) \circ F$,这样首先覆盖了所有 $n>1$ 的排列:$F -x$,而再考虑前述的“切割”部分,如果切割成了 $k$ 段,那么就有 $k-1$ 种被覆盖的方法。也即需要再加上
\begin{align*}
&\quad 2\left(\sum_{k\ge 2} (k-2)H^k\right)\\
& = \frac{2H^3}{(1-H)^2}\\
&= \frac 2{H^{-1}(H^{-1}-1)^2}\\
&= \frac {2F^3}{1+F}
\end{align*}
因此,我们就有
\begin{align*}
(G - x) \circ F &= F - x + \frac {2F^3}{1+F}\\
G \circ F &= 2F + \frac{2F^3}{1+F} - x\\
G &= 2x + \frac{2x^3}{1+x} - F^{\langle -1 \rangle}
\end{align*}
可见,问题关键在于计算 $F$ 的复合逆。

\paragraph{解级数、法 1} 我们知道其困难程度约等于给出 $A$,计算 $P = F\circ A$。对于数列 $f_n = [n\ge 1]n!$,由 $f_n = [n=1] + nf_{n-1}$ 可得一个 ODE:
$$
F = x(1 + F + xF')
$$
根据 $P' = (F'\circ A)\cdot A'$,我们可得
\begin{align*}
F\circ A &= A(1 + F\circ A + A(F'\circ A))\\
P &= A \left(1+ P + A \frac{P'}{A'}\right)
\end{align*}
注意到这个式子通过 $\Theta \left(\frac{n\log^2 n}{\log\log n}\right)$\footnote{目前理论最优为 $\mathcal O\left(n\log n\mathrm{e}^{2\sqrt{\log 2\log\log n}}\right)$} 的半在线卷积来计算的话,等于说是复合就是半在线的,因此我们每次确定了 $F^{\langle -1 \rangle}$ 的前 $m$ 项,计算出此时的 $m+1$ 项,就能知道 $F^{\langle -1 \rangle}$ 的 $m+1$ 项。

\paragraph{解级数、法 2} 我们考虑直接得到 $A = F^{\langle -1 \rangle}$ 的一个微分方程:
\begin{align*}
F\circ A &= x\\
(F'\circ A)\cdot A' &= 1\\
\left[\left(\frac Fx - 1-F\middle ) \right/ x\circ A\right]\cdot A' &= 1\\
\left[ \frac xA - 1 - x \middle]\right/ A \cdot A' &= 1\\
x\cdot A' &= (1+x)AA'+A^2
\end{align*}
由此,我们亦得到了一个可以通过半在线卷积计算的式子。
$$
a_n = -\sum_k (k+1)(a_{k+1}+a_k) a_{n-k}
$$
\emph{需注意半在线卷积的时候当 $v_k$ 确定了,此时还会影响 $u_{k-1}$ 的值。因此递归应当是一个形如 $(l, r]$ 的形式,表示 $l$ 的 $v$ 值已经算出。}

\paragraph{点评}

上述的两种方法最后均是将问题转为计算一个一阶微分方程,那么可否通过其通解在 $\Theta(n\log n)$ 内完成计算呢?在推导过程中我们会发现有一些困难,以法一中的式子为例:

\newpage

\section{Permutation}

\paragraph{简要题意} 对于 $1\le n\le N$,求出所有 $f_n \bmod M$,其中 $f_n$ 表示有多少个 $n$ 阶排列 $p$ 存在 $p_i=i$ 也存在 $p_j=n-j+1$。

保证 $N\le 10^7$ 且 $N+1\le P\le 10^9$。

\paragraph{解法} 设这两个条件是 $P_1,P_2$,那么 $\neg P_1$ 看起来是更自然的,也就是 $\forall i,p_i\neq i$。其中 $\neg P_1$ 和 $\neg P_2$ 都是错排列的数量,我们还要算 $\forall i, (p_i \neq i) \wedge (p_i \neq n-i+1)$,这是问题的关键。

我们对每个 $(i,i)$ 和 $(i,n-i+1)$ 容斥,那么需要共同考虑 $i$ 和 $n-i+1$,其中强制选一个有 $4$ 种方法,强制选两个有 $2$ 种方法。这里我们需要考虑 $n$ 的奇偶性。

\begin{itemize}
\item 对于 $n=2k$,那么方案数即为 $(x^2-4x+2)^k$ 点乘 $\sum_{n\ge 0} n! x^n$。
\item 对于 $n=2k+1$,那么方案数即为 $(x^2-4x+2)^k(x-1)$ 点乘 $\sum_{n\ge 0} n! x^n$。
\end{itemize}

考虑 Euler 积分:
$$
\int_0^\infty x^n \me^{-x} \di x = n!
$$

那么设积分
\begin{align*}
I_n &= \int_0^\infty (x^2-4x+2)^n \me^{-x} \di x\\
J_n &= \int_0^\infty x(x^2-4x+2)^n \me^{-x} \di x
\end{align*}

那么
\begin{itemize}
\item 对于 $n=2k$,方案数即为 $I_k$。
\item 对于 $n=2k+1$,方案数即为 $J_k - I_k$。
\end{itemize}

由分部积分可知,$I_n$ 可进行递推:
\begin{align*}
I_n &= \int_0^\infty (x^2-4x+2)^n \me^{-x}\di x\\
 &= -\int_0^\infty (x^2-4x+2)^n \di (\me^{-x})\\
 &= \left. -(x^2-4x+2)^n \me^{-x} \right |_0^\infty + \int_0^\infty \me^{-x} \di \left[ (x^2-4x+2)^n \right]\\
 &= 2^n + \int_0^\infty n(x^2-4x+2)^{n-1}(2x-4)\me^{-x} \di x\\
 &= 2^n + 2nJ_{n-1} - 4nI_{n-1}
\end{align*}

类似地,$J_n$ 可进行递推:
\begin{align*}
J_n &= \int_0^\infty x(x^2-4x+2)^n \me^{-x} \di x\\
 &= -\int_0^\infty x(x^2-4x+2)^n \di (\me^{-x})\\
  &= \left. -x(x^2-4x+2)^n \me^{-x} \right |_0^\infty + \int_0^\infty \me^{-x} \di \left[ x(x^2-4x+2)^n \right]\\
   &= \int_0^\infty \left[ (x^2-4x+2)^n + nx(x^2-4x+2)^{n-1}(2x-4) \right]\me^{-x} \di x\\
   &= I_n + \int_0^\infty 2n(x^2-4x+2)^{n-1}(x^2-2x)\di x\\
   &= I_n + \int_0^\infty 2n(x^2-4x+2)^{n-1}[(x^2-4x+2)+2x-2]\di x\\
   &= I_n + 2n (I_n + 2J_{n-1} - 2I_{n-1})\\
   &= (2n+1) I_n + 4nJ_{n-1} - 4nI_{n-1}
\end{align*}

由此,整个问题 $\Theta(n)$ 解决。

\paragraph{点评} 对于更一般的问题来说,微分有限性来源于微分有限复合代数幂级数。也就是说对于微分有限的生成函数 $P$,和次数有限的有理分式 $Q$,$Q^n$ 与 $P$ 点乘的结果是关于 $n$ 微分有限的。

\newpage

\section{A story of The Small P}

\paragraph{简要题意} 有正整数 $N,m,k$,问有多少长为 $n(n<N)$ 的序列 $h$,满足 $1\le h_i\le m$,且 $h_i<h_{i+1}$ 的位置恰有 $k$ 个。

同余 $998244353$,保证 $N,m,k<2^{19}-10$ 且 $(N-k+1)m<2^{20}-10$。

\paragraph{解法} 我们首先注意到有 $k$ 个 $<$ 就是说序列被 $\ge$ 分割为了 $n-k$ 段。考查二元 GF 的基本单位,我们需要对每个间隔进行容斥,就有
\begin{align*}
&\quad \sum_n\sum_j\binom{n-1}{j-1} (-1)^{j-1} \binom m n x^nt^j\\
&= t\sum_{n\ge 1} \binom m n x^n (1-t)^{n-1}\\
&= t\frac{ (1+x(1-t))^m -1 }{1-t}
\end{align*}
因此我们就得到了整个序列的 GF,进而求答案:
\begin{align*}
&\quad \sum_{n<N}[x^n][t^{n-k}]\frac 1{1-t\frac{ (1+x(1-t))^m -1 }{1-t}}\\
&= \sum_{n<N}[x^n][t^{n-k}]\frac{1-t}{1-t(1+x(1-t))^m}\\
&= \sum_{n<N}[x^n][t^{n-k}]\sum_j (1-t)t^j(1+x(1-t))^{mj}\\
&= \sum_{n<N}[t^{n-k}]\sum_j \binom {mj}n (1-t)^{n+1}t^j\\
&= \sum_j \sum_{n<N} \binom {mj}n \binom{n+1}{n-k-j}(-1)^{n-k-j}\\
&= \sum_j \sum_{n<N} \binom {mj}n \left[ \binom n{n-k-j} + \binom n{n-k-j-1} \right](-1)^{n-k-j}
\end{align*}
接下来我们考虑抽象出 $A=mj,B=k+j$ 或 $B=k+j+1$,那么就要求和式
$$
\sum_{n<N} \binom An \binom{n}{n-B}(-1)^{n-B}
$$
注意到这就是
$$
\sum_{n<N} \binom AB \binom{A-B}{n-B}(-1)^{n-B}
=\binom AB \binom{A-B-1}{N-B-1}(-1)^{N-B-1}
$$
故而问题已经转化为了 $\Theta(N-k)$ 个组合数。题设刚好保证了组合数的上界较小,故而瓶颈为预处理阶乘,复杂度为 $\Theta((N-k)m)$。

\paragraph{拓展} 实际上由于组合数的下面总是较小的,在 $N,m,k$ 同阶的情况我们总是能够轻易做到 $\tilde O(N)$ 的。

\paragraph{点评} 一个比较简单的通过二元 GF 进行容斥的案例。如果是数排列就会得到一些欧拉数问题。事实上如果我们对答案除以 $m^n$,再取极限 $m \rightarrow \infty$,便会得到欧拉数。

\newpage

\section{Game On a Circle II}

\paragraph{简要题意} 给定长为 $n$ 的排列 $a$ 和概率 $p$。一个指针从 $1$ 开始顺时针走,每次有 $p$ 的概率将当前数删掉加入到序列 $b$ 的末尾。过程直到整个序列被清空,得到排列 $b$。问 $b$ 的\emph{康托展开值}的期望。

同余 $998244353$,保证 $n\le 5\times 10^5$,$\operatorname{ord} (1-p) > n$。

\paragraph{解法}

首先注意到康托展开值是拆成各位之和的,而第 $i$ 位的贡献就是 $| b_i>b_j \land i < j |$ 乘以 $(n-i)!$。

那么一个朴素的想法就是先枚举每个 $a_i$,看看其在序列 $b$ 中出现在位置 $j$ 的概率乘以此时 $| b_i>b_j \land i < j |$ 的期望。根据期望的线性性,我们拆解为对每个 $a_i>a_k$,统计「$a_i$ 在第 $j$ 轮被删去,且此时 $a_k$ 仍存活」的概率。

这里有一个重要的转化,就是将问题转换为一个无穷和进行计算,我们认为一个数被“删掉”是做了至少一次标记,那么设 $q=1-p$,我们可以尝试先写出在第 $k$ 轮 $a_i$ 被删除,此时其对答案的贡献,可以预见这个式子是一个关于 $q^k$ 的多项式,我们求和会得到 $\sum_{k\ge 0} q^{jk} = \frac 1{1-q^j}$,而这种东西的处理是棘手的,所以我们可能务必需要得到 $q^{jk}$ 项系数。具体来讲,对于 $a_i$,如果我们考虑其左边的一个数晚于它被删去,那么在第 $k(\ge 0)$ 轮,可以枚举剩余的数被删去的情况:

$$
\sum_{l=0}^{i-2} \sum_{r=0}^{n-i} \binom {i-2}l \binom{n-i}r (n-l-r-1)! \cdot (1-q^{k+1})^l (1-q^k)^r \cdot q^{(k+1)(i-1-l)} q^{k(n-i-r)} \cdot pq^k
$$

设 $t=q^k$,有

\begin{align*}
& = \sum_{l=0}^{i-2} \sum_{r=0}^{n-i} \binom {i-2}l \binom{n-i}r (n-l-r-1)! \cdot (1-qt)^l (1-t)^r \cdot {(qt)}^{i-1-l} t^{n-i-r} \cdot pt\\
& = p\cdot \sum_{l=0}^{i-2} \sum_{r=0}^{n-i} \binom {i-2}l \binom{n-i}r (n-l-r-1)! \cdot (1-qt)^l (1-t)^r \cdot q^{i-1-l} t^{n-l-r}
\end{align*}

我们将 $(n-l-r-1)!$ 代换成 $u^{n-l-r-1}$,那么这个式子可以直接根据二项式定理合并。

\begin{align*}
& = p\cdot \sum_{l=0}^{i-2} \sum_{r=0}^{n-i} \binom {i-2}l \binom{n-i}r u^{n-l-r-1} \cdot (1-qt)^l (1-t)^r \cdot q^{i-1-l} t^{n-l-r}\\
& = p u^{n-1} q^{i-1} t^n\cdot \sum_{l=0}^{i-2} \sum_{r=0}^{n-i} \binom {i-2}l \binom{n-i}r u^{-l-r} \cdot (1-qt)^l (1-t)^r \cdot q^{-l} t^{-l-r}\\
& = p u^{n-1} q^{i-1} t^n \left( 1 + \frac{1-qt}{uqt} \right)^{i-2} \left( 1 + \frac{1-t}{ut} \right)^{n-i}\\
& = pq u t^2 \left( uqt + 1-qt \right)^{i-2} \left( ut + 1-t \right)^{n-i}\\
& = pq u t^2 (1+qt(u-1))^{i-2} (1+t(u-1))^{n-i}
\end{align*}

我们令 $v^{\bullet}=u(u-1)^{\bullet}$,也就有

$$
= pq t^2 (1+qtv)^{i-2} (1+tv)^{n-i}
$$

我们又注意到可以令 $t^2(tv)^\bullet = x^\bullet$,也就有

$$
= pq (1+qx)^{i-2}(1+x)^{n-i}
$$

我们要对所有的 $i$ 都求出,因此转置问题就是计算 $\sum_i a_i pq (1+qx)^{i-2}(1+x)^{n-i}$,这是显然可以 $\Theta(n\log n)$ 的,不同的变形方法都可以得到结果。

对于另一半,计算可得就是 $p (1+qx)^{i-2}(1+x)^{n-i}$,少一个 $q$。

数点用树状数组就可以做了,因此整个问题复杂度为 $\Theta(n\log n)$。

\paragraph{点评} 原本的思路是先从一个简化的问题出发,然后得到了一种组合转化方法,而我们借助的哑演算法则是让代数推导更加自由,也能够直接从代数推导上得到同样的性质。

\newpage

\section{monster}

\paragraph{简要题意} 有正整数 $k$ 以及 $a_i, p_i$,每回合按 $1\sim k$ 的顺序检查,每次 $a_i$ 有 $p_i$ 的概率减少 $1$,求 $a_1$ 最先减到 $0$ 的概率。

同余 $998244353$,保证 $k\le 5, S=\sum a_i \le 5\times 10^6$,且 $p_i$ 随机生成。

\paragraph{解法}

我们不妨直接将答案写作一个无穷和式,不妨设游戏刚好在第 $n+1$ 个回合结束,那么依此求和就有
$$
\sum_{n\ge 0} \binom n{a_i-1} p_1^{a_1} (1-p_1)^{n+1-a_1} \prod_{i=2}^k
\left[\sum_{j=0}^{a_i-1} \binom n j p_i^j(1-p_i)^{n-j}\right]
$$
这个和式有一个很重要的观察,它本质上是形如 $\sum_{n\ge 0} f(n) q^n$ 的形式,其中 $f$ 是一个 $d=S - k$ 次多项式,而 $q = \prod (1-p_i)$。

但首先我们要确认关于 $f$ 的信息中何者比较好求。事实上我们发现 $f$ 的点值最为好求,我们只需先按照原式的定义进行计算,最后的 $\prod$ 项关于 $n$ 有一个简单的递推关系,这就导出了前 $S$ 项点值的一个 $\Theta(kS)$ 算法。

为了得到最优的复杂度,我们需要考虑基于点值的推导,考虑变换为二项式,即存在系数 $g_i$ 满足
$$
f(n) = \sum_{i=0}^{d} g_i \binom n i
$$
那么我们首先需要考虑对于任何一个 $i$,单纯的 $\binom n i$ 的答案,这是实际上就是负二项式:
$$
\sum_{n\ge 0} \binom n i q^n = \frac{q^i}{(1-q)^{i+1}}
$$
而点值和二项式之间还相差一个反演
$$
g_i = \sum_j \binom i j (-1)^{i-j} f(j)
$$
我们欲求出对于 $0\le n\le d$ 的每个 $n$,我们求出的点值 $f(n)$ 需要点乘的系数,也即
\begin{align*}
&\quad \sum_{i=n}^{d}\binom i n (-1)^{i-n} \frac{q^i}{(1-q)^{i+1}}\\
&= \frac 1{1-q}\sum_{i=n}^{d}\binom i n (-1)^{i-n} t^i, \qquad t=\frac q{1-q}\\
&= \frac 1{1-q}\sum_{i=n}^{d} [x^n] q^i(x-1)^i\\
&= \frac 1{1-q}[x^n] \frac{[q(x-1)]^{d+1}-1}{q(x-1)-1}
\end{align*}
由此,这一和式可以直接通过多项式的除法解决,这一部分复杂度为 $\Theta(S)$。

综上,我们就得到了一个 $\Theta(kS)$ 的算法。

\paragraph{点评} 在考虑复杂度的时候,我们常常需要考虑以下几类信息的转化:连续点值、下降幂系数、多项式系数、插值点值……其中,在只求一个答案的时候问题尤为特殊,因为这时基于的一个转换可能类似于计算 $\mathbf {uM}\mathbf v^{\mathsf T}$,先计算何者就有可能影响复杂度了。

\newpage

\section{Game on Tree}

\paragraph{简要题意}

一颗二叉树,划分成 $A,B,L$ 三部分,其中 $L$ 为叶节点,其余点均有两个孩子。每个叶节点有 $c,d$ 值,一组策略为给每个点选择一个重子,称双方的收益为从根沿重子走到的对应 $c,d$ 值。称一个策略\emph{纳什均衡}当且仅当任取 $A$ 中策略改变均不使得 $c$ 更大,任取 $B$ 中策略改变均不使 $d$ 更大。对所有 $\forall u\in L,1\le c_u,d_u\le k$ 的情况,对纳什均衡的策略数求和。对每个节点的子树求出答案。

同余 $p$。保证 $n\le 5000, k,p\le 10^9$。

\paragraph{解法}

首先一个简单的想法是考虑一个朴素的 DP,$F_u(i, j)$ 表示 $u$ 所在的子树达到纳什均衡,其中 $A$ 的最优策略达到了 $i$,$B$ 的最优策略达到了 $j$。我们来先看一下假设 $u\in A$ 的情况。其 $u\in B$ 的转移是完全对称的:

设两子为 $l, r$。若以 $l$ 为重子,注意此时 $l$ 的子树维持纳什均衡,但 $r$ 的子树只需保证修改 $A$ 可达的最大值 $\le i$。我们记 $X_r(i)$ 表示最大值 $\le i$ 的情况。那么就有
$$
F_u(i,j) \leftarrow F_l(i,j) \cdot X_r(i)
$$
对称地,我们可得此时有
$$
F_u(i,j) = F_l(i,j) \cdot X_r(i) + F_r(i,j) \cdot X_l(i)
$$
而此时的 $X,Y$ 呢?对于本层的 $X$,两个子树均能取到,故有
$$
X_u(i) = 2X_l(i)\cdot X_r(i)
$$
而对于 $Y$($B$ 的最优化)来说,这被 $A$ 完全摆布,另一子树的细节就没用了,所以是
$$
Y_u(i) = Y_l(i) \cdot 2^{|A_r|+|B_r|}k^{2|L_r|} + Y_r(i) \cdot 2^{|A_l|+|B_l|}k^{2|L_l|}
$$
不难看出 $X,Y$ 是多项式,但 $F$ 是个二元多项式。然而经过归纳,我们发现 $F$ 的两元是实质上独立的!我们考虑归纳,若
$$
F_u = U_u \cdot V_u
$$
有良定义的关系
$$
\begin{aligned}
X_u(i) &= 2^{|A_u|} k^{|L_u|} U_u(i)\cdot i\\
Y_u(i) &= 2^{|B_u|} k^{|L_u|} V_u(i)\cdot i
\end{aligned}
$$
我们进行归纳。在叶节点处,令 $U_u(i)=V_u(i)=1$ 即可。还是考虑 $A$ 位置。
\begin{align*}
F_u(i,j) &= F_l(i,j) \cdot X_r(i) + F_r(i,j) \cdot X_l(i)\\
&= U_l(i,j)V_l(i,j)U_r(i,j)i 2^{|A_r|} k^{|L_r|} +U_r(i,j)V_r(i,j)U_l(i,j)i 2^{|A_l|} k^{|L_l|}\\
U_u(i,j)V_u(i,j)&= U_l(i,j)U_r(i,j) i\left( V_l(i,j) 2^{|A_r|} k^{|L_r|}+ V_r(i,j) 2^{|A_l|} k^{|L_l|} \right)\\
U_u(i,j)&=U_l(i,j)U_r(i,j) i\\
V_v(i,j)&=V_l(i,j) 2^{|A_r|} k^{|L_r|}+ V_r(i,j) 2^{|A_l|} k^{|L_l|}\\
X_u(i)&=2 X_l(i)\cdot X_r(i)\\
&=2 \left(2^{|A_l|} k^{|L_l|} U_l(i)\cdot i\right) \left(2^{|A_r|} k^{|L_r|} U_r(i)\cdot i\right)\\
&=2^{|A_u|}k^{|L_u|} U_l(i)U_r(i)i^2\\
&=2^{|A_u|}k^{|L_u|} U_u(i)\cdot i\\
Y_u(i)&= Y_l(i) \cdot 2^{|A_r|+|B_r|}k^{2|L_r|} + Y_r(i) \cdot 2^{|A_l|+|B_l|}k^{2|L_l|}\\
&= \left( V_l(i) \cdot 2^{|A_r|}k^{|L_r|} + V_r(i) \cdot 2^{|A_l|}k^{|L_l|} \right)\cdot 2^{|B_l|+|B_r|} k^{|L_u|}\\
&= Y_l(i) \cdot 2^{|A_r|}k^{|L_r|} + Y_r(i) \cdot 2^{|A_l|}k^{|L_l|} 
\end{align*}
故归纳成立,因此我们只需维护 $U,V$ 即可。

我们只需维护 $0\sim n$ 位置的点值,然后通过插值即可得到 $k$ 位置的结果。

综上,本题在 $\Theta(n^2)$ 复杂度内得到解决。

\paragraph{拓展} 求某一棵树答案的时候,注意到这一关系是两个孩子的乘法或线性组合,我们可以对表达式树通过分治在 $\Theta(n\log ^2 n)$ 内完成计算。然后藉由伯努利数即可在 $\Theta(n\log n)$ 内求出答案。

进一步地,我们通过转置原理,可以先 $\Theta(n\log n)$ 预处理每个 $x^j$ 的前缀和,从而在 $\Theta(n\log^2n)$ 内求出所有点的答案。

\paragraph{点评} 其独立性似乎有一些组合意义上的理解,但在代数形式上似乎没有那么显然。

\newpage

\section{小 Y 的序列}

\paragraph{简要题意}

给出正整数 $n, w$,对于每个 $1\le k\le n$,求出
$$
\sum_{\{1\le a_i\le w\}_{i=1}^n} \# \left\{
i \middle \vert \exists i_1<i_2<\dots <i_k, a_{i_j} < a_i
\right \}
$$
同余 $998244353$,保证 $1\le n\le 114514, 1\le w\le 114514191$。

\paragraph{解法}

首先考察题设的 $\exists$ 条件,我们不如先数出最大的 $k$,然后做一个后缀和即可。

先转成随机生成然后数期望,我们接下来分别考虑答案的 GF,当 $a_{i+1}=j+1$ 的时候,前面每个数有 $\frac jw$ 的概率给 $k$ 加一。
\begin{align*}
&\quad \frac 1w\sum_{i=0}^{n-1} \sum_{j=0}^{w-1} \left(\frac {w-j}w + \frac jw x\right)^i\\
&= \frac 1w\sum_{i=0}^{n-1} \sum_{j=0}^{w-1} \left(1 - \frac jw + \frac jw x\right)^i\\
&= \left[\frac 1w\sum_{i=0}^{n-1} \left(1 - t + t x\right)^i\right] \cdot \left\{ t^k \rightarrow \sum_{j=0}^{w-1} \left(\frac j w\right)^k \right\}\\
&= \left[\frac 1w \cdot \frac{1 - \left(1 - t + t x\right)^n}{1 -  \left(1 - t + t x\right)} \right] \cdot \left\{ t^k \rightarrow \sum_{j=0}^{w-1} \left(\frac j w\right)^k \right\}\\
&= \left[\frac 1w \cdot \frac{1 - \left(1 - t + t x\right)^n}{t(1-x)} \right] \cdot \left\{ t^k \rightarrow \sum_{j=0}^{w-1} \left(\frac j w\right)^k \right\}\\
&= \frac 1{w(1-x)} \cdot \left[\frac{1 - \left(1 - t + t x\right)^n}{t} \right] \cdot \left\{ t^k \rightarrow \sum_{j=0}^{w-1} \left(\frac j w\right)^k \right\}\\
&= \frac 1{w(1-x)} \cdot \left[\frac{1 - \left(1 - t\right)^n}{t} - \sum_{j=1}^n \binom n j (1-t)^{n-j}t^{j-1} x^j \right] \cdot \left\{ t^k \rightarrow \sum_{j=0}^{w-1} \left(\frac j w\right)^k \right\}
\end{align*}
其中,可以先通过伯努利数将 $\{\cdot \}$ 部分的所有求和结果算出。接着 $[ \cdot ]$ 无非是考虑 $(1-t)^{n-j}t^{j-1}$,这无非还是一次转置原理的简单应用。

综上,本题在 $\Theta(n\log n)$ 时间内得到解决。

\paragraph{点评} 若使用上文演示的哑元推导,这道题还是没有什么特别的难点可言的。

\newpage

\section{tree \& prime}

\paragraph{简要题意}

一颗 $n$ 个节点的树,每个节点 $i$ 经过时有 $p_i$ 的概率得到一枚金币。对每个节点 $i$,问有多大的概率从 $1$ 号点走到其时,得到的金币数量为素数。

同余 $998244353$,保证 $1\le n\le 10^5$。

\paragraph{解法}

素数的条件看起来就是比较一般的,我们首先将问题一般化为给定数列 $a_i$,总共有 $i$ 个金币时得到 $a_i$ 的权值,问每个点处权值的期望。如此一来问题就是找到一个高效算法计算
$$
\mathbf v = \mathbf {Ma}
$$
我们不妨考虑这个问题的$\displaystyle{\text{\emph{转}} \atop \text{\emph{置}}}$,即先考虑给定一个 $\mathbf v$,如何计算
$$
\mathbf a = \mathbf M^{\mathsf T} \mathbf v
$$
我们考虑其意义,这相当于有 $v_i$ 的概率以 $i$ 节点为终点,然后 $a_i$ 为最终获得 $i$ 个金币的概率。用多项式乘法来刻画就有
$$
f_i(x) = (1-p_i + p_ix) \left(v_i + \sum_{(i,j)\in T} f_j(x)\right)
$$
最终的 $\mathbf a$ 就是 $f_1(x)$ 的系数向量。注意到过程中多项式的次数就是子树的高度,我们可以在长链剖分上进行分治。即在过程中维护积与和 $p, s$,有
\begin{align*}
p_o & \leftarrow p_l \cdot p_r\\
s_o & \leftarrow s_l \cdot p_r + s_r
\end{align*}
在递归的尾部,需注意 $s$ 要加上辅树的所有结果。本算法的复杂度为 $\Theta(n\log ^2n)$。接着我们将算法转置回去,就只需倒写过程,变为
\begin{align*}
s_r & \leftarrow \mathbf{MUL}(p_r)^{\mathsf T} s_o \\
s_l & \leftarrow s_o
\end{align*}
并且递归到叶子后还需将答案贡献给辅树的子问题。需注意过程中保留的多项式次数由转置前算法的每一步次数确定。复杂度为 $\Theta(n\log ^2n)$。

\paragraph{点评} 转置原理的朴素应用。

\newpage

\section{Yet Another Permutation Problem}

\paragraph{简要题意}

问对于所有长为 $n$ 的排列,有多少排列存在一个连续上升段 $\ge k$。对所有 $k$ 回答,对输入的大质数取模。

保证 $1\le n\le 1000, 10^8\le M\le 10^9+9$。

\paragraph{解法}

首先经过转化,只需要数所有连续段均 $<k$ 的即可。容易想到容斥,我们可以先假设有一些基本单位,然后容斥掉它们拼起来的情况,可得基本单位是 $\frac{x-x^k}{1-x}$,其容斥单位为
\begin{align*}
&\quad 1-\frac 1{1+\frac{x-x^k}{1-x}}\\
&= 1-\frac{1-x}{1-x^k}\\
&= \frac{x-x^k}{1-x^k}\\
&= \sum_{j\ge 0} \left(x^{jk+1} - x^{(j+1)k}\right)
\end{align*}
而排列实际的拼接是 EGF 的形式,因此答案为
$$
n![x^n] \frac 1{\displaystyle1 - \sum_{j\ge 0} \left( \frac{x^{jk+1}}{(jk+1)!}  - \frac{x^{(j+1)k}}{[(j+1)k]!}\right)}
$$
由于下面的项数有 $\Theta(n/k)$ 项,因此暴力求逆的复杂度为 $\Theta(n^2/k)$,求所有的答案是 $\Theta(n^2\log n)$。

\paragraph{越过数据范围}

注意到我们实际上有两个算法,一个算法是 $\Theta(n\log n)$ 的求逆,一个算法是 $\Theta(n^2/k)$ 的暴力求逆。那么我们显然应该选择当前最快的算法来跑。对于 $k\le \frac n{\log n}$ 的时候是 $\Theta(n\log n)$ 更快一点,所以对于前 $\frac n{\log n}$ 项,总共复杂度为 $\Theta(n^2)$。而对于后面的来说,根据调和级数更加精确的分析:$H_n = \ln n + \gamma + O(1/n)$,因此从 $\frac n{\log n}$ 到 $n$ 求和,大约是 $\log n - \log(n/\log n) = \log\log n$,因此将这两个算法混在一起使用,复杂度就降到了 $\Theta(n^2\log\log n)$。

\paragraph{越过 $\Omega(n^2)$}

这个问题的 $\Omega(n^2)$ 的 bound,是不可突破的吗?

我们刚刚实际上不仅求出了 $k=1\sim n$,还求出了更小的 $n$ 的所有答案。但实际上我们并没有必要这么做。现在我们不妨假设有一个算法,他能在 $\Theta((n/k)^c)$ 的复杂度内计算一个 $k$ 对应的答案。容易发现在任何 $k$ 的位置,$\Theta(n^2/k)$ 的算法都是不占优的,因为若 $k>\frac n{\log n}$,那么 $(n/k)^c < (\log n)^c \ll n < n^2/k$。因此我们只需在 $k\le n^{1-1/c}(\log n)^{-1/c}$ 的部分使用 $\Theta(n\log n)$ 的算法,剩下的部分使用 $\Theta((n/k)^c)$ 的算法。估界有
\begin{align*}
&\quad \sum_{k>L} \left(\frac nk\right)^c\\
&< \int_L^\infty n^cx^{-c} \,\mathrm dx\\
&= \left.\frac 1{1-c}n^c x^{1-c}\right|_{x=L}^\infty\\
&= \frac 1{c-1}n^cL^{1-c},\quad L=n^{1-1/c}(\log n)^{-1/c}\\
&= \frac 1{c-1} n^{2-1/c}(\log n)^{1-1/c}
\end{align*}
此时,复杂度两部分均是 $\Theta(n^{2-1/c}(\log n)^{1-1/c})$。

回顾我们要求的,无非是
$$
n![x^n] \frac 1{1-x-x^{k+1}A(x^k)-x^kB(x^k)}
$$
经过这样的改写,我们不难考虑枚举三个部分的次数,显然这等价于如此求和:
$$
\sum_{u,v,w\ge 0} \binom{u+v+w}{u,v,w} U^uV^vW^w 
$$
而我们提取系数 $[x^n]$,这就固定了 $x$ 项的次数,也就有
\begin{align*}
&\quad [x^n] \frac 1{1-x-x^{k+1}A(x^k)-x^kB(x^k)}\\
&= [x^n]\sum_{u,v,w\ge 0} \binom{u+v+w}{u,v,w}x^{u(k+1)+vk+w}A(x^k)^uB(x^k)^v\\
&= [x^n]\sum_{u,v,w\ge 0} \binom{u+v+w}{u,v,w} [x^{n-u(k+1)-vk-w}] A(x^k)^uB(x^k)^v
\end{align*}
此时 $A(x^k)^uB(x^k)^v$ 只有 $k$ 的倍数次才有值,因此维护和求值都是只关于 $n/k$ 的。

如使用暴力卷积,那么复杂度为 $\Theta((n/k)^4)$,总复杂度是 $\Theta(n^{7/4}(\log n)^{3/4})$。

如这里使用 FFT,那么复杂度为 $\Theta((n/k)^3\log (n/k))$,总复杂度是 $\Theta(n^{5/3}\log n)$。

\paragraph{点评} 这个问题可能还有一条路,熟悉者可能会认为有一定前途,那就是用 DFT(单位根反演)将分母表为 $\mathrm{e}^{\omega x}$ 之和。但经验上说,诸如 $\tan x, \frac x{\mathrm e^x-1}$ 之类的多项式均未有优于多项式求逆的方法,因此这条思路究竟能带来什么不同的做法还有待考量。

\newpage

\section{Communication Network}

\paragraph{简要题意}

给一颗 $n$ 个节点的树 $T_1$,要求枚举 $n^{n-2}$ 种树,求和

$$
\sum_{T_2} |T_1 \cap T_2|2^{|T_1 \cap T_2|}
$$

同余 $998244353$,保证 $1\le n\le 2\times 10^6$。

\paragraph{解法}

设 $t=2x+2$,这道题无非求算
$$
[x^1] \sum_{T_2} t^{|T_1 \cap T_2|}
$$
考虑一个经典的 Pr\"ufer 序列结果,连通块大小分别为 $a_1,\dots,a_k$ 的图连成树的方案数是
$$
n^{k-2}\prod_{i=1}^k a_i
$$
其中的乘积项可以认为是 $[y^1](1+y)^{a_i}$,因此我们可以通过树形 DP 进行容斥来计算。注意到因为我们只求 $[x^1]$,所以可以 $\bmod x^2$,故 DP 过程维护的信息量皆为常数。复杂度为 $\Theta(n)$。

\paragraph{点评} 约等于「WC2019 数树」的一个子任务。

\end{document}
